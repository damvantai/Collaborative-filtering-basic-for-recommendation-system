\chapter{Kinh nghiệm}
Ta sử dụng ngôn ngử lập trình python và các thư viện cơ bản trong tính toán khoa học trong python.

Về phần hệ khuyến nghị ta sử dụng thư viện graphlab 

Còn phần thực hiện FunkSVD ta sử dụng thư viện pyRecLab (Recommendation lab for Python)
\section{Datasets}
Kiểm tra được thực hiện trong hai tập dữ liệu Movielens được cung cấp bởi GroupLens. Tập dữ liệu Movielens bao gồm đánh giá của người xem cho phim. Hai tập đó có kích thước khác nhau, độc lập với nhau. Tập dữ liệu được sử dụng trong báo cáo là 100k phát hành năm 1988 và 1M được phát hành năm 2003. Cả hai tập dữ liệu bao gồm các người xem đánh giá ít nhất 20 phim. Nó bao gồm 100000 đánh giá từ 943 người xem trong 1682 phim với mật độ 6.37\%. Tham số đưa ra kết quả tốt nhất trong tập Movielens 100k sau đó được đưa vào tập 1M để đánh giá độ chính xác của thuật toán. Tập dữ liệu 1M bao gồm 1 million đánh giá từ 6040 người sử dụng trong 3706 bộ phim với mật độ 4.47\% . Cả hai tập dữ liệu có đánh giá từ 1 đến 5.

Lý do sử dụng hai tập dữ liệu Movielens là bởi vì chúng được sử dụng rãi trong nhiều báo cáo. Họ cũng sử dụng quy ước đánh giá. Cả hai tập dữ liệu là cùng tỉ lệ đánh giá tương ứng. Kích thước hai bộ dữ liệu không quá lớn có thể chạy nhanh ra kết quả.

\begin{table}[H]
\caption{Movielens dataset properties}
\label{table:2}
\begin{center}
\begin{tabular}{|c|c|c|c|c|c|}
\hline 
Name & Users & Movies & Ratings-Scale & Ratings & Density \\ 
\hline 
ML 100k & 943 & 1682 & 1-5 & 100.000 & 6.30\% \\ 
\hline 
ML 1M & 6040 & 3706 & 1-5 & 1.000.209 & 4.47\% \\ 
\hline 
\end{tabular} 
\end{center}
\end{table}
